%%%%%%%%%%%%%%%%%%%%%%%%%%%%%%%%%%%%%%%%%
% This is the LaTeX template of Jonas Schultheiss. It'll be used for the IPA and VA.
% 
% This template can be downloaded here:
% https://github.com/
% 
% This template is heavily inspired by Marco Roth's template:
% https://github.com/marcoroth
%
% Template license:
% MIT
% https://github.com/jonasschultheiss/LaTeX-Template/blob/master/LICENSE
% ./LICENSE
%%%%%%%%%%%%%%%%%%%%%%%%%%%%%%%%%%%%%%%%%

%%%%%%%%%%%%%%%%%%%%%%
% Document & Variables
%%%%%%%%%%%%%%%%%%%%%%

\newcommand{\version}{0.0.1}
\newcommand{\docdate}{\today}
\newcommand{\docname}{EcoHub Testautomation und Mapping}
\newcommand{\compiledfilename}{ipa\_kp\_pax\_2025.pdf}
\newcommand{\task}{Individualle praktische Arbeit} % e.g. Modul 152/IPA/...
\newcommand{\taskname}{Entwicklung einer Testautomation für die EcoHub-Plattform und Implementierung von Mapping-Funktionen}
\newcommand{\clientlogopath}{./images/bbzbl.png}
\newcommand{\kandidat}{Kento Puleo}
\newcommand{\docauthor}{\kandidat}


%%%%%%%%%%
% Packages
%%%%%%%%%%

\usepackage[
  a4paper,
  left=20mm,
  right=20mm,
  top=35mm,
  bottom=35mm,
  headheight=35mm
]{geometry}
\usepackage[utf8]{inputenc}
\usepackage[babel,english]{csquotes}
\usepackage[english]{babel}
\usepackage{acronym}
\usepackage[section]{placeins}
\usepackage[Conny]{fncychap}
\usepackage[T1]{fontenc}
\usepackage[T1]{fontenc}
\usepackage[table]{xcolor}
\usepackage{caption}
\usepackage{color}
\usepackage{fancyhdr}
\usepackage{graphics}
\usepackage{graphicx}
\usepackage{helvet}
\usepackage{lastpage}
\usepackage{lipsum}
\usepackage{listings}
\usepackage{multicol}
\usepackage{multirow}
\usepackage{parskip}
\usepackage{pdfpages}
\usepackage{pifont}
\usepackage{tabularx}
\usepackage{longtable}
\usepackage{tocloft}
\usepackage{xurl}
\usepackage{wrapfig}
\usepackage{xcolor}
\usepackage{afterpage}
\usepackage[backend=biber, style=numeric]{biblatex}
\usepackage{suffix}
\usepackage{ragged2e}
\usepackage{float}
\usepackage[
  left = \flqq{},% 
  right = \frqq{},% 
  leftsub = \flq{},% 
  rightsub = \frq{} %
]{dirtytalk}

%%%%%%%%
% Colors
%%%%%%%%

\definecolor{lightgray}{RGB}{224,224,224}
\definecolor{darkgray}{RGB}{192,192,192}

\renewcommand{\listfigurename}{List of figures}


%%%%%%%%%%%%%%%%%%%
% Headers & Footers
%%%%%%%%%%%%%%%%%%%

\pagestyle{fancy}
\renewcommand{\chapterpagestyle}{fancy}
\renewcommand{\partpagestyle}{fancy}
\fancyhead{}
\fancyhead[L]{\fontsize{10}{12}\selectfont\textbf{\task} \\ \fontsize{8}{10} \selectfont \taskname \\ \kandidat}
\fancyhead[R]{\includegraphics[width=6cm]{\clientlogopath}}
\fancyfoot{}
\fancyfoot[L]{\fontsize{10}{11} \selectfont \compiledfilename}
\fancyfoot[C]{\fontsize{9}{11} \docdate}
\fancyfoot[R]{\fontsize{10}{11} \selectfont Page \thepage\space of \computelastpage}

\renewcommand{\headrulewidth}{0pt}

\makeatletter
\newif\if@mainmatter
\renewcommand{\frontmatter}{%
  \clearpage
  \pagenumbering{Roman}
  \edef\computelastpage{%
    \uppercase{\romannumeral\numexpr\getpagerefnumber{LastFrontPage}-1\relax}}}
\renewcommand{\mainmatter}{%
  \clearpage
  \immediate\write\@auxout{\noexpand\newlabel{LastFrontPage}{{}{\arabic{page}}}}%
  \@mainmattertrue
  \pagenumbering{arabic}
  \def\computelastpage{\pageref{LastPage}}}
\makeatother


%%%%%%
% Font
%%%%%%

\renewcommand{\familydefault}{\sfdefault}

%%%%%
% TOC
%%%%%

\renewcommand{\cftchapafterpnum}{\vspace{5pt}}

%%%%%
% LOF
%%%%%

\captionsetup{justification=centering}

%%%%%%%%%%
% Graphics
%%%%%%%%%%

\setcounter{totalnumber}{5}
\ChNumVar{\Huge}
\ChTitleVar{\huge\sffamily}
\ChNameVar{\large\sffamily}
\ChRuleWidth{0.5pt}

%%%%%%%%%%%%%%
% Line spacing
%%%%%%%%%%%%%%

\renewcommand{\baselinestretch}{1,3}

%%%%%%%%%%%%%%%%
% PDF formatting
%%%%%%%%%%%%%%%%

\definecolor{codegray}{gray}{0.9}
\newcommand{\code}[1]{\colorbox{codegray}{\texttt{#1}}}
\usepackage[
  bookmarks=true,           % Lesezeichen erzeugen
  bookmarksopen=true,       % Lesezeichen ausgeklappt
  bookmarksnumbered=true,   % Anzeige der Kapitelzahlen
  breaklinks=true,          % Ermöglicht einen Umbruch von URLs
  colorlinks=true,          % Einfärbung von Links
  linkcolor=black,          % Linkfarbe: schwarz
  anchorcolor=black,        % Ankerfarbe: schwarz
  citecolor=black,          % Literaturlinks: schwarz
  filecolor=black,          % Links zu lokalen Dateien: schwarz
  menucolor=black,          % Acrobat Menü Einträge: schwarz
  urlcolor=black,           % URL-Farbe: schwarz
  pdftitle={\docname},
  pdfauthor={\kandidat},
  pdfsubject={\docname},
  pdfkeywords={Wirtschaft, Recht, law, economics, busniness, BBZ-BL, bbz, bl, bbzbl, \kandidat}
]{hyperref}

%%%%%%%%%%%%%%
% Reset indent
%%%%%%%%%%%%%%

\setlength{\parindent}{0pt}

%%%%%%%%%
% Columns
%%%%%%%%%

\setlength{\columnseprule}{1pt}
\def\columnseprulecolor{\color{black}}

%%%%%%%%%%%%%
% Titel fonts
%%%%%%%%%%%%%

\newcommand{\titlesize}{\fontsize{25pt}{20pt}\selectfont}
\newcommand{\subtitlesize}{\fontsize{15pt}{10pt}\selectfont}
\newcommand{\subsubtitlesize}{\fontsize{10pt}{5pt}\selectfont}

\newcommand{\amk}[1]{\flqq #1\frqq}